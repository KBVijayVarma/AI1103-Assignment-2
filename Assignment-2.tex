\documentclass[journal,12pt,twocolumn]{IEEEtran}

\usepackage{setspace}
\usepackage{gensymb}
\singlespacing
\usepackage[cmex10]{amsmath}

\usepackage{amsthm}

\usepackage{mathrsfs}
\usepackage{txfonts}
\usepackage{stfloats}
\usepackage{bm}
\usepackage{cite}
\usepackage{cases}
\usepackage{subfig}

\usepackage{longtable}
\usepackage{multirow}

\usepackage{enumitem}
\usepackage{mathtools}
\usepackage{steinmetz}
\usepackage{tikz}
\usepackage{circuitikz}
\usepackage{verbatim}
\usepackage{tfrupee}
\usepackage[breaklinks=true]{hyperref}
\usepackage{graphicx}
\usepackage{tkz-euclide}

\usetikzlibrary{calc,math}
\usepackage{listings}
    \usepackage{color}                                            %%
    \usepackage{array}                                            %%
    \usepackage{longtable}                                        %%
    \usepackage{calc}                                             %%
    \usepackage{multirow}                                         %%
    \usepackage{hhline}                                           %%
    \usepackage{ifthen}                                           %%
    \usepackage{lscape}     
\usepackage{multicol}
\usepackage{chngcntr}

\DeclareMathOperator*{\Res}{Res}

\renewcommand\thesection{\arabic{section}}
\renewcommand\thesubsection{\thesection.\arabic{subsection}}
\renewcommand\thesubsubsection{\thesubsection.\arabic{subsubsection}}

\renewcommand\thesectiondis{\arabic{section}}
\renewcommand\thesubsectiondis{\thesectiondis.\arabic{subsection}}
\renewcommand\thesubsubsectiondis{\thesubsectiondis.\arabic{subsubsection}}


\hyphenation{op-tical net-works semi-conduc-tor}
\def\inputGnumericTable{}                                 %%

\lstset{
%language=C,
frame=single, 
breaklines=true,
columns=fullflexible
}
\begin{document}

\newcommand{\BEQA}{\begin{eqnarray}}
\newcommand{\EEQA}{\end{eqnarray}}
\newcommand{\define}{\stackrel{\triangle}{=}}
\bibliographystyle{IEEEtran}
\raggedbottom
\setlength{\parindent}{0pt}
\providecommand{\mbf}{\mathbf}
\providecommand{\pr}[1]{\ensuremath{\Pr\left(#1\right)}}
\providecommand{\qfunc}[1]{\ensuremath{Q\left(#1\right)}}
\providecommand{\sbrak}[1]{\ensuremath{{}\left[#1\right]}}
\providecommand{\lsbrak}[1]{\ensuremath{{}\left[#1\right.}}
\providecommand{\rsbrak}[1]{\ensuremath{{}\left.#1\right]}}
\providecommand{\brak}[1]{\ensuremath{\left(#1\right)}}
\providecommand{\lbrak}[1]{\ensuremath{\left(#1\right.}}
\providecommand{\rbrak}[1]{\ensuremath{\left.#1\right)}}
\providecommand{\cbrak}[1]{\ensuremath{\left\{#1\right\}}}
\providecommand{\lcbrak}[1]{\ensuremath{\left\{#1\right.}}
\providecommand{\rcbrak}[1]{\ensuremath{\left.#1\right\}}}
\theoremstyle{remark}
\newtheorem{rem}{Remark}
\newcommand{\sgn}{\mathop{\mathrm{sgn}}}
\providecommand{\abs}[1]{\vert#1\vert}
\providecommand{\res}[1]{\Res\displaylimits_{#1}} 
\providecommand{\norm}[1]{\lVert#1\rVert}
%\providecommand{\norm}[1]{\lVert#1\rVert}
\providecommand{\mtx}[1]{\mathbf{#1}}
\providecommand{\mean}[1]{E[ #1 ]}
\providecommand{\fourier}{\overset{\mathcal{F}}{ \rightleftharpoons}}
%\providecommand{\hilbert}{\overset{\mathcal{H}}{ \rightleftharpoons}}
\providecommand{\system}{\overset{\mathcal{H}}{ \longleftrightarrow}}
	%\newcommand{\solution}[2]{\textbf{Solution:}{#1}}
\newcommand{\solution}{\noindent \textbf{Solution: }}
\newcommand{\cosec}{\,\text{cosec}\,}
\providecommand{\dec}[2]{\ensuremath{\overset{#1}{\underset{#2}{\gtrless}}}}
\newcommand{\myvec}[1]{\ensuremath{\begin{pmatrix}#1\end{pmatrix}}}
\newcommand{\mydet}[1]{\ensuremath{\begin{vmatrix}#1\end{vmatrix}}}
\numberwithin{equation}{subsection}
\makeatletter
\@addtoreset{figure}{problem}
\makeatother
\let\StandardTheFigure\thefigure
\let\vec\mathbf
\renewcommand{\thefigure}{\theproblem}
\def\putbox#1#2#3{\makebox[0in][l]{\makebox[#1][l]{}\raisebox{\baselineskip}[0in][0in]{\raisebox{#2}[0in][0in]{#3}}}}
     \def\rightbox#1{\makebox[0in][r]{#1}}
     \def\centbox#1{\makebox[0in]{#1}}
     \def\topbox#1{\raisebox{-\baselineskip}[0in][0in]{#1}}
     \def\midbox#1{\raisebox{-0.5\baselineskip}[0in][0in]{#1}}
\vspace{3cm}
\title{Assignment 2}
\author{Vijay Varma - AI20BTECH11012}
\maketitle
\newpage
\bigskip
\renewcommand{\thefigure}{\theenumi}
\renewcommand{\thetable}{\theenumi}

%
Download latex-tikz codes from 
%
\begin{lstlisting}
https://github.com/KBVijayVarma/AI1103-Assignment-2
\end{lstlisting}
\section{Problem}
(GATE EC, Q. 12) 

P and Q are considering to apply for a job. The probability that P applies for the job is $\frac{1}{4}$, the probability that P applies for the job given that Q applies for the job is $\frac{1}{2}$, and the probability that Q applies for the job given that P applies for the job is $\frac{1}{3}$. Then the probability that P does not apply for the job given that Q does not apply for the job is

(A) $\frac{4}{5}$    (B) $\frac{5}{6}$    (C) $\frac{7}{8}$    (D) $\frac{11}{12}$

\section{Solution}
Let A represent the event that P applies for the job. Let B represent the event that Q applies for the job. 

According to the given information in the question,
\begin{align}
\pr{A} = \frac{1}{4} \\
\pr{A|B} = \frac{1}{2} \\
\pr{B|A} = \frac{1}{3} 
\end{align}
According to the definition of Conditional Probability,
\begin{align}
\pr{A|B} = \frac{\pr{AB}}{\pr{B}} \label{2.0.4} \\
\pr{B|A} = \frac{\pr{AB}}{\pr{A}} \label{2.0.5}
\end{align}
On substituting the values of $\pr{A}$, $\pr{B|A}$ in \eqref{2.0.5}, we get
\begin{align}
\frac{1}{3} = \frac{\pr{AB}}{\frac{1}{4}} \\
\pr{AB} = \left(\frac{1}{3} \right) \left(\frac{1}{4} \right) \\
\therefore \pr{AB} = \frac{1}{12}
\end{align}
Now substituting the values of $\pr{A|B}$, $\pr{AB}$ in \eqref{2.0.4}, we get
\begin{align}
\frac{1}{2} = \frac{\frac{1}{12}}{\pr{B}} \\
\pr{B} = \frac{\left(\frac{1}{12} \right)}{\left(\frac{1}{2} \right)} \\
\therefore \pr{B} = \frac{1}{6}
\end{align}
The probability that P does not apply for the job given that Q does not apply for the job is given by $\pr{A^{'}|B^{'}}$.

Now,
\begin{align}
A^{'}B^{'} = (A + B)^{'} \\
\Rightarrow \pr{A^{'}B^{'}} = \pr{(A + B)^{'}} \\
\therefore \pr{A^{'}B^{'}} = 1 - \pr{A + B} \label{2.0.14}
\end{align}
As we know that,
\begin{align}
\pr{A+B} = \pr{A} + \pr{B} - \pr{AB} \label{2.0.15}
\end{align}
By substituting the values of $\pr{A}$, $\pr{B}$, $\pr{AB}$ in \eqref{2.0.15}, we get
\begin{align}
\pr{A+B} = \frac{1}{4} + \frac{1}{6} - \frac{1}{12} \\
= \frac{3 + 2 - 1}{12} \\
\therefore \pr{A+B} = \frac{1}{3}
\end{align}
$\therefore$ Required Probability is 
\begin{align}
\pr{A^{'}|B^{'}} = \frac{\pr{A^{'}B^{'}}}{\pr{B^{'}}} \label{2.0.19}   
\end{align}
By substituting the values of $\pr{B}$, $\pr{A^{'}B^{'}}$ [from \eqref{2.0.14}] in \eqref{2.0.19}, we get
\begin{align}
\pr{A^{'}|B^{'}} = \frac{1 - \pr{A + B}}{1 - \pr{B}} \\
\pr{A^{'}|B^{'}} = \frac{1 - \left(\frac{1}{3} \right)}{1 - \left(\frac{1}{6} \right)} \\
\Rightarrow \pr{A^{'}|B^{'}} = \frac{\left(\frac{2}{3} \right)}{\left(\frac{5}{6} \right)} = \frac{4}{5}\\
\end{align}
$\therefore$ $\pr{A^{'}|B^{'}}$ = $\frac{4}{5}$ = 0.8

Hence, the probability that P does not apply for the job given that Q does not apply for the job is equal to $\frac{4}{5}$

$\therefore$ The correct option is (A) $\frac{4}{5}$ .


\end{document}